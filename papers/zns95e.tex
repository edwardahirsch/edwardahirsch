\pagewidth{108mm}
\pageheight{185mm} % 113 x 185 with 20.5pt headline	%AS
% redefinitions:
\let\le\leqslant
\let\ge\geqslant

\def\varinjlim{\mathop{\vtop{\ialign{##\crcr
 \hfil\lat\rm lim\hfil\crcr\noalign{\nointerlineskip}\rightarrowfill\crcr
 \noalign{\nointerlineskip\kern-\ex@}\crcr}}}}
\def\varprojlim{\mathop{\vtop{\ialign{##\crcr
 \hfil\lat\rm lim\hfil\crcr\noalign{\nointerlineskip}\leftarrowfill\crcr
 \noalign{\nointerlineskip\kern-\ex@}\crcr}}}}
\def\varliminf{\mathop{\underline{\vrule height\z@ depth.2exwidth\z@
 \hbox{\lat\rm lim}}}}
\def\varlimsup{\mathop{\overline{\hbox{\lat\rm lim}}}}

\def \Bbbone{\roman{1\mathchoice{\kern-0.25em}{\kern-0.25em}
	{\kern-0.2em}{\kern-0.2em}I}}
\def\RM#1{\leavevmode\skip@\lastskip\unskip\/%
        \ifdim\skip@=\z@\else\hskip\skip@\fi{\rm#1}}

\LimitsOnInts
\TagsOnRight
\tolerance=1700
\hfuzz=1.5pt
\frenchspacing
\let\mz=\cprime

\LimitsOnInts
\TagsOnRight
\tolerance=1700
\hfuzz=1.5pt
\frenchspacing
\let\mz=\cprime


% ***********************************************************************

%\input amstex
%\documentstyle{amsppt}
%\input nologo.sty
%\parindent = 0pt
\topmatter
\title
%\nofrills
On a construction of symbolic realization of hyperbolic
automorphisms of torus
\endtitle
\rightheadtext{Hyperbolic automorphisms of torus}
\leftheadtext{E.~A.~.Hirsch}

%\rightheadtext{�� ����� �������樨 }
%\af Saint-Petersburg State \newline
%University
%��⥬�⨪�-��堭��᪨� 䠪����, \\
%�.-������, 198904, �����
%\endaf
%\da ����㯨�� 15 ���� 1995 �.
%\endda
\author E.~A.~Hirsch\,${}^\maltese$ \endauthor
\thanks
${}^\maltese$ Supported by 
Russian Foundation for Fundamental Research grant \,94\;-\;01\;-\;00921
\endthanks
\endtopmatter

\document
%\Russian

In the paper [1] A.~M.~Vershik suggested a general approach to constructing
of arithmetic isomorphism of hyperbolic automorphisms of torus and symbolic
shifts. An initial step of suggested scheme consisted in the following
conjecture.

   Let $T$ be an automorphism of torus; we shall use the same notation for the
corresponding transformation of ${\Bbb Z}^n$. Let a vector $v \in {\Bbb Z}^n$
have the property that its orbit under the automorphism $T$ in ${\Bbb Z}^n$
is infinite. Let us denote by $G$ the semigroup generated by the orbit of the
element $v$ under $T$.

\proclaim{Conjecture}  There is a natural number $N$ such that every element
$g$ of the semigroup $G$ can be represented in the following way:
$$g=\sum_{k \in {\Bbb Z}}\,e_k(g)\,T^kv\,,$$
where $e_k(g)$ is a finite sequence of numbers $0,1,\ldots,N$.
\endproclaim

 There exists a proof of this conjecture in case when the characteristic
polynomial of $T$ is of the form $x^n-a_{n-1}x^{n-1}-\ldots -a_1x_1-1$, where
$a_{n-1}\ge\ldots \ge a_2\ge a_1\ge 0$
(see [2]).
A similiar approach was realized in [3] for an automorphism such that the
dominant root
of its characteristic polynomial is a Pisot number. (But instead of
the conjecture under discussion a slightly different statement was used.)
A.~M.~Vershik suggested a hypothesis that this result can be extended
to a wider class of automorphisms.
\!\!\footnote" * "{As A.~M.~Vershik informed the author, he together with
R.~Kenyon proved that the Conjecture will be true if one replace the condition
$e_k(g)\in\{0,1,\ldots,N\}$ with the condition
$e_k(g)\in S$, where $S$ is a finite subset of the Galois field of the
characteristic polynomial of the automorphism $T$.}
In this note we prove that it is impossible to extend it to a certain class
of automorphisms. Namely, we consider automorphisms,
characteristic polynomials of which have nonnegative coefficients and at
least two roots of different modulus outside the unit circle. Thus in case of
nonnegative coefficients two opportunities remain unconsidered:
\smallskip
\roster
\item"(1)" if there are roots on the unit circle;
\item"(2)" if all roots outside the unit circle have the same modulus.
\endroster

\medskip
So let us consider a polynomial $p\,(z)=z^n-a_{n-1}z^{n-1}-\ldots -a_1z_1-1$,
where $a_{n-1}, a_{n-2}, \ldots ,a_1\ge 0,\;\; \sum a_i^2\,>\,0$.

Let $S$ be the set of all two-sided fi\-ni\-te se\-qu\-en\-ces\linebreak
$(\ldots ,b_{-2},b_{-1},b_0,b_1,b_2,\ldots )$
of nonnegative integers. Elements of $S$ will be regarded as formal sums of
the form
$$
\sum_{i=s_1}^{s_2}\,b_i\,z^i\;\;\;  (s_2\ge
s_1).\;\;\;\;\;\;\;\;\;\;\;\;\;\;\;\;
$$

Let $v\in S$. We denote the largest of its coefficients by {\bf $rk\,v$}.

Let $v,w\in S$. We write {\bf $v\backsim_p w$}, if
$\exists \,t\in {\Bbb Z}: p \;|\; (v-w)\,z^t$.

%\medpagebreak
We call $p$ {\bf decomposing}, if
$$
\exists C \in {\Bbb N} \;\,\forall v \in S \;\,\exists w \in S\;\;
(v\backsim_p w \and rk\,w\le C).
$$

We note that $p$ has exactly one positive root $r>1$.

\proclaim{Theorem}
If the polynomial $p$ satisfying the conditions above has a (complex) root
$q$ which is situated outside the unit disk and differs by modulus from
$r$, then $p$ is not decomposing.
\endproclaim

\demo{Proof}
Suppose the assertion does not hold, i.e. $p$ is decomposing.
Then
$$
\exists C \in {\Bbb N} \;\,\forall v \in S \;\,\exists w \in S\;\;
(v\backsim_p w \and rk\,w\le C).
$$

Take $v=A$ (constant).
$w \backsim_p A$, hence $w\,(r)=w\,(q)=A$.
Let
$$
w=\sum_{i=s_1}^{s_2}\,b_iz^i, \;\; b_{s_2}>0.
$$
Then $A=w\,(r)\,\ge\,r^{s_2}$, i.e. $s_2 \le \log_rA$.

\smallskip
From the other hand,
$$
|A|=|w\,(q)|< C\,\frac{|q|^{s_2+1}}{|q|-1},
$$
i.e.
$$
s_2> \log_{|q|}A + \log_{|q|}\frac{|q|-1}{C}-1.
$$
The value of $\log_{|q|}\frac{|q|-1}{C}-1$ is a constant not depending on
$A$; let us denote it by $D$.

\medskip
So
$$
\log_rA\ge s_2>\log_{|q|}A+D.
\tag{$*$}
$$

However, notice that $r>|q|$, since otherwise, i.e. if $r<|q|$,
we have
 $$
\gather
0=\frac{p\,(q)}{(q)^n}
=1-\sum_{i=1}^{n-1}\,a_i\,q^{-i}-q^{-n}\ge\\
\ge
1-\sum_{i=1}^{n-1}\,a_i\,|q|^{-i}-|q|^{-n}=\frac{p\,(|q|)}{|q|^n}>
\frac{p\,(r)}{|q|^n}=0
\endgather
$$ (contradiction).

Thus for sufficiently large $A$ the inequality ($*$) does not hold and this
contradicts to the initial hypothesis that $p$ is decomposing
%$\qed$
\enddemo


\Refs
\ref
\no1
\by A.~M.~Vershik,
\paper  The arithmetic isomorphism of hyperbolic automorphisms of torus and
sophic shifts
\jour Functional analysis and its applications
\yr 1992
\vol 26, No.~3
\pages 22--27
\endref

\ref
\no2
\by Ch.~Frougny and B.~Solomyak,
\paper Finite
beta-expansions
\paperinfo Ergodic Theory  Dynamical
Systems, {\bf 12} (1992), 713--723
\endref

\ref
\no3
\by  A.~Bertrand-Mathis,
\paper Developpement en base $\theta$;
repartition modulo un de la suite $(x\theta^n)_{n\ge 0}$;
langages codes et $\theta$-shift
\jour Bull. Soc. math. France
\vol 114
\yr 1986
\pages 271--323
\endref
\endRefs

\enddocument
